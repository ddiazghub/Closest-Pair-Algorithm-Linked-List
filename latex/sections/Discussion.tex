\section{Análisis de Resultados}

En la presente sección se realizará un análisis de los resultados obtenidos de la parte experimental del laboratorio que se presentaron en la sección anterior. Se verificará que estos resultados concuerden con lo esperado para validar el funcionamiento de ambos algoritmos. La gráfica de los resultados del algoritmo de fuerza bruta se puede ver en la figura 3. Lo que se puede destacar principalmente acerca de estos resultados es un incremento de el número de iteraciones que realizar el algoritmo que aumenta de forma cuadrática cuando incrementa el tamaño del conjunto de datos de entrada. El tiempo transcurrido incrementa de manera cuadrática similar al número de iteraciones. Lo que se puede destacar de estos resultados es que son muy similares a los del la implementación utilizando arraylists. Por otro lado, la figura 4 contiene la gráfica de los resultados del algoritmo recursivo. Para este algoritmo se puede ver que el número de iteraciones que se realizan incrementa en un orden más similar con el linear al incrementar el tamaño del conjunto de datos de entrada. Estas iteraciones incluyen el proceso de dividir recursivamente el conjunto de datos de entrada en varios subconjuntos, ejecutar el algoritmo de fuerza bruta entre los elementos de cada uno de estos conjuntos y de ejecutar el algoritmo de fuerza bruta entre los elementos de 2 diferentes conjuntos que pueden ser la pareja más cercana. Para conjuntos de datos muy pequeños, la diferencia con respecto al algoritmo de fuerza bruta es despreciable, sin embargo, cuando el tamaño del conjunto de datos incrementa, el algoritmo recursivo presenta un desempeño mucho mejor al del algoritmo de fuerza bruta al reducirse el número de comparaciones que se hacen por cada uno de los elementos a una o dos dependiendo del tamaño del subconjunto de datos. Además de lo anterior, al compararse con la implementación realizada con ArrayLists, se puede apreciar que existe una diferencia en el número de comparaciones del algoritmo. En la gráfica del algoritmo recursivo se puede apreciar que el número de comparaciones es mayor en la implementación con listas enlazadas que en la implementación con arraylists, lo anterior se puede justificar en que las listas enlazadas no tienen una forma de acceso aleatorio, por lo cual cuando se quiera acceder a un elemento que no está en la cabeza ni en el final de la lista, es necesario recorrerla secuencialmente, lo anterior puede ralentizar el algoritmo, especialmente en el proceso de particionar la lista, donde es necesario dividir la lista por la mitad hasta que se tenga listas muy pequeñas.